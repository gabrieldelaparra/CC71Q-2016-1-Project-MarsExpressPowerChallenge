\documentclass[../Main.tex]{subfiles}

\begin{document}

\section{Propuesta de trabajo}
Debido al tamaño de los datos, se considera poco viable analizar toda la información en conjunto. Lo anterior principalmente debido a que al intentar visualizar todas las mediciones de los sistemas térmicos, se tuvo problemas en la capacidad computacional del equipo en donde se realizaba la tarea (i7-4600m Quad-2.9GHz, 8GB RAM).
\newline \par
Se considera entonces la opción de realizar segmentación de la información por rangos según los eventos visibles en cuanto a la recepción de luz. A partir de esta segmentación se propone buscar la superposición entre los datos para verificar si existen relaciones y posteriormente procesar los datos de forma más depurada.
\newline \par
Considerando lo anterior, se propone realizar las siguientes tareas:
\begin{itemize}
	\item Preparar la información de consumo energético para realizar clustering. El proposito de esto será encontrar los sistemas que sean de calefacción/refrigeración y así también separar idealmente los 33 circuitos térmicos según los 7 sensores que lleva la nave.
    \item Preparar y segmentar la información de incidencia solar y compararla con la información sobre eclipses, distancias con el sol, dirección de la nave y comandos enviados.
\end{itemize}

Lo anterior permitirá generar una estimación de relación de los datos. Se entiende que esta estimación puede no ser correcta debido al poco conocimiento específico del dominio del problema. Debido a lo anterior, se propone iterar con secciones de los datos para buscar la validación de las relaciones.
\newline \par
Finalmente se introduciran los datos a WEKA o R y se probará con distintos algoritmos para buscar una clasificación adecuada y construir un modelo que pueda enviado al concurso para su evaluación.

\end{document}